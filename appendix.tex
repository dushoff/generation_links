
\subsection{Comparison with the SEIR model}

The SEIR (susceptible-exposed-infectious-recovered) model is given by
\begin{equation}
\begin{aligned}
\frac{dS}{dt} &= - \beta SI/N\\
\frac{dE}{dt} &= \beta SI/N - \sigma E\\
\frac{dI}{dt} &= \sigma E - \gamma I\\
\frac{dR}{dt} &= \gamma I
\end{aligned}
\end{equation}
It is assume that latent and infectious periods are exponentially distributed with mean $1/\sigma$ and $1/\gamma$, respectively.

For this model, the generation interval distribution can be written as follows \cite{Sven07}:
\begin{equation}
g({\tau}) = \frac{\sigma \gamma}{\sigma - \gamma} \left(\exp (-\gamma \tau) - \exp (- \sigma \tau) \right).
\end{equation}
The mean generation interval is
\begin{equation}
\bar{G} = \frac{1}{\sigma} + \frac{1}{\gamma}.
\end{equation}
and the squared coefficient of variation is
\begin{equation}
\kappa = 1 - \frac{2}{\bar{G}^2 \sigma \gamma}
\end{equation}

\cite{lipsitch2003transmission, roberts2007model} derived the following \rR\ relationship for the SEIR model:
\begin{equation}
\RR = 1 + r\left(\frac{1}{\sigma} + \frac{1}{\gamma} \right) + \frac{r^2}{\sigma \gamma}.
\end{equation}
Reparameterizing, we have 
\begin{equation}
\begin{aligned}
\RR &= 1 + r \bar{G} + \frac{(1-\kappa) r^2 {\bar{G}}^2}{2}\\
&= 1 + \rho + \frac{(1-\kappa)}{2} \rho^2
\end{aligned}
\end{equation}
Note that Taylor expansion of \eref{genExp} yields the following:
\begin{equation}
\tsup{\RR}{gamma} \approx 1 + \rho + \frac{(1-\kappa)}{2} \rho^2 + O(\rho^3) \label{eq:taylor},
\end{equation}
where $\tsup{\RR}{gamma}$ is an \RR\ estimate based on the gamma approximation.
When $\sigma = \gamma$, $g(\tau)$ follows a gamma distribution and $O(\rho^3)$ vanishes.
We expect the gamma approximation to work well when average duration of latent and infectious periods are similar.

\pagebreak
\subsection{Ebola example}

\begin{figure}[htbp] \centering
	\includegraphics[width=1.0\textwidth]{link_calculations/ebola_gamma.pdf}
\caption{
We perform the same analysis as we did in \sref{EbolaEx} assuming gamma distributed incubation and infectious periods. 
We find that the gamma approximated speed-strength relationship matches the true relationship almost perfectly in this case.
Once again, we adjust the standard deviation of the incubation period to match the reported coefficient of variation in serial interval distributions. Rest of the parameters and points as in \fref{EbolaCurve}
}
	\label{fig:ebolaGamma}
\end{figure}

\pagebreak
\subsection{Normal approximation}

\begin{figure}[htbp] \centering
	\includegraphics[width=1.0\textwidth]{link_calculations/ebola_normal.pdf}
\caption{
Approximating generation-interval distributions with a normal distribution has two problems. First, the distribution extends to negative values, which are biologically impossible. Second, as a consequence, the normal approximation predicts a saturating and eventually a decreasing $r-\RR$ relationship for large $r$. Parameters and points as in \fref{EbolaCurve}.
}
	\label{fig:ebolaNormal}
\end{figure}

\pagebreak
\subsection{Robustness of the gamma approximation}

The moment-matching method (approximating $\RR$ based on estimated mean and variance of the generation interval) has an appealing simplicity, and works well for all of the actual disease parameters we tested (the breakdown for rabies distributions occurs for values of $\RR$ well above observed values). We therefore wanted to compare its robustness given small sample sizes along with that of the more sophisticated maximum likelihood method. \fref{ebolaSample} shows results of this experiment. When sample size is limited, estimates using MLE tend to be substantially close to the known true values in these experiments. 
As we increase sample size, our estimates become narrower. 
We also find that using the gamma approximated speed-strength relationship gives narrower estimates than the two naive estimates even when the sample size is extremely small ($n = 10$).
It is important to note that \fref{ebolaSample} only conveys uncertainty in the estimate of coefficient of variation of generation interval distributions.
Estimation of mean generation interval will introduce additional uncertainty into estimates of the reproduction number. 

\begin{figure}[htbp] \centering
	\includegraphics[width=1.0\textwidth]{link_calculations/ebola_sample.pdf}
\caption{
%
The effect of small sample size on approximated relationship between $r$ and \RR.
(black solid curve) The relationship between growth rate and \RR~using a known generation-interval distribution (see \fref{EbolaCurve}).
(colored curves) Estimates based on finite samples from this distribution: dashed curves show the median and solid curves show 95\% quantiles of 1000 sampling experiments.
Note that the upper 95\% quantile of the moment approximation and MLE approximation overlap.
(dotted curves) Naive approximations based on exponential (lower) and fixed (upper) generation distributions.
%
}
	\label{fig:ebolaSample}
\end{figure}
