\documentclass[12pt, landscape]{article}

\usepackage[margin=0.5in]{geometry}

\usepackage{lmodern}
\usepackage{amssymb,amsmath}

\usepackage[colorinlistoftodos]{todonotes}
\usepackage[colorlinks=true, allcolors=blue]{hyperref}
\urlstyle{same}  % don't use monospace font for urls
\bibliographystyle{vancouver}

\newcommand{\rR}{\mbox{$r$--$\cal R$}}
\newcommand{\RR}{\ensuremath{{\cal R}}}
\newcommand{\Rx}[1]{\ensuremath{{\cal R}_{#1}}} 
\newcommand{\Ro}{\Rx{0}}
\newcommand{\Reff}{\Rx{\mathit{eff}}}
\newcommand{\Tc}{\ensuremath{C}}

\newcommand{\eref}[1]{(\ref{eq:#1})}
\newcommand{\fref}[1]{Fig.~\ref{fig:#1}}
\newcommand{\Fref}[1]{Fig.~\ref{fig:#1}}
\newcommand{\sref}[1]{Sec.~\ref{#1}}
\newcommand{\frange}[2]{Fig.~\ref{fig:#1}--\ref{fig:#2}}
\newcommand{\tref}[1]{Table~\ref{tab:#1}}
\newcommand{\tlab}[1]{\label{tab:#1}}

\newcommand{\comment}[3]{\textcolor{#1}{\textbf{[#2: }\textit{#3}\textbf{]}}}
\newcommand{\jd}[1]{\comment{cyan}{JD}{#1}}
\newcommand{\swp}[1]{\comment{magenta}{SWP}{#1}}
\newcommand{\dc}[1]{\comment{blue}{DC}{#1}}
\newcommand{\jsw}[1]{\comment{red}{JSW}{#1}}
\newcommand{\hotcomment}{}

\usepackage{lscape}

\usepackage{graphicx,grffile}

\usepackage{booktabs}% http://ctan.org/pkg/booktabs
\newcommand{\tabitem}{~~\llap{\textbullet}~~}
\newcommand{\tabphant}{\hphantom\tabitem}

\setlength{\emergencystretch}{3em}  % prevent overfull lines


\begin{document}

\begin{table}
\centering
\footnotesize
\begin{tabular}{l|l|l|c}
\hline
\bf Method & \bf Required information & \bf Assumptions/Notes & \bf References \\
\hline
Incidence based & \tabitem Generation interval CDF & \tabitem Uses cumulative incidence directly & \cite{nishiura2010correcting} \\
 & \tabitem Incidence reports & \tabphant without estimating the initial growth rate  & \\
 & & \tabitem Entire generation interval distribution is required to obtain CDF & \\
\hline
Delta approximation & \tabitem Mean generation interval & \tabitem Assumed fixed generation interval & \cite{wallinga2007generation, roberts2007model} \\
\hline
Empirical (histogram-based) & \tabitem Generation interval samples & \tabitem Relies on binning generation intervals & \cite{wallinga2007generation} \\
& & \tabphant into discrete histograms &\\
\hline
Empirical (sample-based) & \tabitem Generation interval samples & \tabitem Requires more samples than & \cite{hampson2009transmission}\\
& & \tabphant the histogram-based method & \\
\hline
Euler-Lotka & \tabitem Entire generation interval distribution & \tabitem Uses moment generating function & \cite{Lotka} \\
\hline
Normal approximation & \tabitem Mean/SD generation interval & \tabitem Assuems normally distributed generation intervals & \cite{wallinga2007generation}\\
& & \tabitem Predicts decreasing \rR\ relationship for high $r$ & \\
\hline
SIR model & \tabitem Mean generation interval & \tabitem Assumes exponentially distributed generation interval & \cite{anderson1992infectious, pybus2001epidemic, ferguson2005strategies, wallinga2007generation} \\
& & \tabitem Predicts linear relationship between $r$ and \RR\ & \\
\hline
SEIR (gamma) & \tabitem Mean/SD latent period & \tabitem Assumes gamma distributed latent and infectious periods & \cite{anderson1980spread, wearing2005appropriate, wallinga2007generation, roberts2007model} \\
& \tabitem Mean/SD infectious period & & \\
\hline
SEIR (exponential) & \tabitem Mean generation interval & \tabitem Assumes exponentially distributed latent and infectious periods & \cite{lipsitch2003transmission, chowell2007comparative, roberts2007model} \\
& \tabitem Ratio of the infectious period to & \tabitem Predicts quadratic relationship between $r$ and \RR\ & \\
& \tabphant the generation interval & & \\
\hline
Subexponential & \tabitem Generation interval shape & \tabitem Assumes sub-exponential growth rate & \cite{chowell2016characterizing} \\
& \tabitem Corresponding generation interval parameters & \tabitem Provides examples for uniform, gamma, exponential and delta &\\
& \tabitem Sub-exponential growth paramete & \tabphant distributed generation intervals & \\
& \tabitem Disease generation & & \\
\hline
Trapezoid approximation & \tabitem Period of no infection after exposure & \tabitem Assumes trapezoid shaped infection kernel & \cite{roberts2007model} \\
& \tabitem Time until maxium infectivity& & \\
& \tabitem Duration of maximum infectivity & & \\
& \tabitem Time until recovery & &\\
\hline
\end{tabular}
\end{table}

\clearpage
\pagebreak

\bibliography{table}

\end{document}
