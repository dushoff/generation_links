

\subsection{Normal approximation}


\begin{figure}[htbp] \centering
	\includegraphics[width=1.0\textwidth]{Generation_distributions/EbolaCurve.normal.flat.Rout.pdf}
\caption{
Approximating a generation intervals distribution with a normal distribution can be problematic as it supports negative values, which are biologically impossible. As a consequence, the normal approximation (red curve) eventually predicts a decreasing $r-\RR$ relationship for large $r$.
}
	\label{fig:ebolaNormal}
\end{figure}

\subsection{Robustness of the gamma approximation}

The moment-matching method (approximating $\RR$ based on estimated mean and variance of the generation interval) has an appealing simplicity, and works well for all of the actual disease parameters we tested (the breakdown for rabies distributions occurs for values of $\RR$ well above observed values). We therefore wanted to compare its robustness in statistical estimation to that of the more sophisticated maximum likelihood method. \fref{ebolaSample} shows results of this experiment. When sample size is limited, estimates using MLE tend to be substantially close to the known true values in these experiments. However, it is worth noting that using the observed moment gives narrower estimates than the two naive estimates.

\begin{figure}[htbp] \centering
	\includegraphics[width=1.0\textwidth]{Generation_distributions/ebola_sample_curve.Rout.pdf}
\caption{
%
The effect of sample size on estimates of \RR.
(black) the relationship between growth rate and \RR~using a known generation-interval distribution (see \fref{EbolaCurve}).
(colors) estimates based on finite samples from this distribution: solid curves show the median of 1000 sampling experiments, and shading shows the range where 95\% of the results fall.
Blue shows estimates based on estimated mean and CV.
Red shows estimates based on maximum likelihood fits.
%
}
	\label{fig:ebolaSample}
\end{figure}

