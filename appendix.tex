
\documentclass[12pt,]{article}
\usepackage{lmodern}
\usepackage{amssymb,amsmath}

\usepackage[colorinlistoftodos]{todonotes}
\usepackage[colorlinks=true, allcolors=blue]{hyperref}
\urlstyle{same}  % don't use monospace font for urls
\usepackage{biblatex}

\newcommand{\RR}{\ensuremath{{\cal R}}}
\newcommand{\Rx}[1]{\ensuremath{{\cal R}_{#1}}} 
\newcommand{\Ro}{\Rx{0}}
\newcommand{\Reff}{\Rx{\mathit{eff}}}
\newcommand{\Tc}{\ensuremath{C}}

\newcommand{\eref}[1]{(\ref{eq:#1})}
\newcommand{\fref}[1]{Fig.~\ref{fig:#1}}
\newcommand{\sref}[1]{Sec.~\ref{#1}}
\newcommand{\frange}[2]{Fig.~\ref{fig:#1}--\ref{fig:#2}}
\newcommand{\tref}[1]{Table~\ref{tab:#1}}
\newcommand{\tlab}[1]{\label{tab:#1}}

\newcommand{\comment}[3]{\textcolor{#1}{\textbf{[#2: }\textit{#3}\textbf{]}}}
\newcommand{\jd}[1]{\comment{cyan}{JD}{#1}}
\newcommand{\swp}[1]{\comment{magenta}{SWP}{#1}}
\newcommand{\dc}[1]{\comment{blue}{DC}{#1}}
\newcommand{\jsw}[1]{\comment{red}{JSW}{#1}}

\addbibresource{refs.bib}
\usepackage{graphicx,grffile}

\setlength{\emergencystretch}{3em}  % prevent overfull lines

\title{Exploring how generation intervals link strength and speed of epidemics - Appendix}
\author{Sang Woo Park \and David Champredon \and Joshua Weitz \and Jonathan Dushoff}
\date{July 2017}

\begin{document}
\maketitle

\subsection{Normal approximation}

\subsection{Robustness of the gamma approximation}

The moment-matching method (approximating $\RR$ based on estimated mean and variance of the generation interval) has an appealing simplicity, and works well for all of the actual disease parameters we tested (the breakdown for rabies distributions occurs for values of $\RR$ well above observed values). We therefore wanted to compare its robustness in statistical estimation to that of the more sophisticated maximum likelihood method. \fref{ebolaSample} shows results of this experiment. When sample size is limited, estimates using MLE tend to be substantially close to the known true values in these experiments.

\begin{figure}[htbp] \centering
	\includegraphics[width=1.0\textwidth]{Generation_distributions/ebola_sample_curve.Rout.pdf}
\caption{
%
The effect of sample size on estimates of \RR.
(black) the relationship between growth rate and \RR~using a known generation-interval distribution (see \fref{EbolaCurve}).
(colors) estimates based on finite samples from this distribution: curves show the median of 1000 sampling experiments, and shading shows the range where 95\% of the results fall.
Blue shows estimates based on estimated mean and CV.
Red shows estimates based on maximum likelihood fits.
%
}
	\label{fig:ebolaSample}
\end{figure}

\printbibliography

\end{document}

