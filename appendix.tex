
\setcounter{figure}{0}
\renewcommand{\thefigure}{S\arabic{figure}}

\subsection{Normal approximation}

\begin{figure}[htbp] \centering
	\includegraphics[width=1.0\textwidth]{link_calculations/ebola_gamma.pdf}
\caption{
Fitting the Ebola model to an alternative, gamma-based pseudo-realistic distribution. SWP to add more text.
}
	\label{fig:ebolaNormal}
\end{figure}

\subsection{Normal approximation}

\begin{figure}[htbp] \centering
	\includegraphics[width=1.0\textwidth]{link_calculations/ebola_normal.pdf}
\caption{
Approximating generation-interval distributions with a normal distribution has two problems. First, the distribution extends to negative values, which are biologically impossible. Second, as a consequence, the normal approximation predicts a saturating and eventually a decreasing $r-\RR$ relationship for large $r$. Parameters and points as in \fref{EbolaCurve}.
}
	\label{fig:ebolaNormal}
\end{figure}

\subsection{Robustness of the gamma approximation}

The moment-matching method (approximating $\RR$ based on estimated mean and variance of the generation interval) has an appealing simplicity, and works well for all of the actual disease parameters we tested (the breakdown for rabies distributions occurs for values of $\RR$ well above observed values). We therefore wanted to compare its robustness given small sample size ($n=100$) along with that of the more sophisticated maximum likelihood method. \fref{ebolaSample} shows results of this experiment. When sample size is limited, estimates using MLE tend to be substantially close to the known true values in these experiments. However, it is worth noting that using the observed moment gives narrower estimates than the two naive estimates even when the sample size is very small.

\begin{figure}[htbp] \centering
	\includegraphics[width=1.0\textwidth]{link_calculations/ebola_sample.pdf}
\caption{
%
The effect of small sample size on approximated relationship between $r$ and \RR.
(black solid curve) The relationship between growth rate and \RR~using a known generation-interval distribution (see \fref{EbolaCurve}).
(colored curves) Estimates based on finite samples from this distribution: dashed curves show the median and solid curves show 95\% quantiles of 1000 sampling experiments.
(dotted curves) Naive approximations based on exponential (lower) and fixed (upper) generation distributions.
%
}
	\label{fig:ebolaSample}
\end{figure}




