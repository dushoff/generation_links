\documentclass[12pt]{article}
\usepackage[utf8]{inputenc}

\usepackage{color}

\usepackage{lmodern}
\usepackage{amssymb,amsmath}

\newcommand{\rR}{\mbox{$r$--$\cal R$}}
\newcommand{\RR}{\ensuremath{{\cal R}}}
\newcommand{\RRhat}{\ensuremath{{\hat \cal R}}}
\newcommand{\Rx}[1]{\ensuremath{{\cal R}_{#1}}} 
\newcommand{\Ro}{\Rx{0}}
\newcommand{\Reff}{\Rx{\mathit{eff}}}
\newcommand{\Tc}{\ensuremath{C}}

\newcommand{\rev}{\subsection*}
\newcommand{\revtext}{\textsf}
\setlength{\parskip}{\baselineskip}
\setlength{\parindent}{0em}

\newcommand{\comment}[3]{\textcolor{#1}{\textbf{[#2: }\textsl{#3}\textbf{]}}}
\newcommand{\jd}[1]{\comment{cyan}{JD}{#1}}
\newcommand{\swp}[1]{\comment{magenta}{SWP}{#1}}
\newcommand{\dc}[1]{\comment{blue}{DC}{#1}}
\newcommand{\jsw}[1]{\comment{green}{JSW}{#1}}
\newcommand{\hotcomment}[1]{\comment{red}{HOT}{#1}}

\begin{document}

\noindent Dear Editor:

Thank you for the chance to revise this submission. Below please find our responses from the previous round.

\rev{Reviewer 1}

\revtext{This revision is substantial, being approximately four pages longer than the
original, and with seventeen more references. The title has changed on the
manuscript but remains the same on the cover sheet. The authors should
standardise on ‘reproduction number’, which is correct, instead of ‘reproductive
number’, or at least be consistent. See Figure 1. They should also use ‘infection’
rather than ‘disease’ in many places.}

Sorry about the cover sheet. The revision was substantial largely in response to good criticisms in the previous round.
We have changed `reproductive number' to `reproduction number' and replaced `disease' with `infection' in appropriate places.

\revtext{Table 1. Suggested additional reference Roberts \& Nishiura 2011. PLoS One 6: e17835.}

We have added the reference to the table.

\revtext{Table 1. typo row 8 column 3.}

Fixed.

\revtext{Page 6 “taking only non-negative values makes it more biologically realistic than the normal”. What does this mean? The normal only takes positive values too.}

We mean that the domain of a normal distribution extends from negative infinity to positive infinity.
We have made this clearer in the current text: ``restricting the domain to only non-negative values makes it more biologically realistic than the normal.''

\revtext{Page 6 “its theoretical properties and practical importance have not yet been explored in depth.” This is nonsense. There are whole books about the gamma function and its properties are taught to undergraduates.}

We have changed the text to clarify this point: ``its theoretical and practical importance in explaining the \rR\ relationship has not yet been explored in depth''

\revtext{Page 8, line 4. “converges to exp”}

We have followed the reviewer's comment in changing the text.

\revtext{Page 8, paragraph “Characterizing . . . higher R” could be a lot clearer. So could the next paragraph.}

We have clarified these paragraphs as follows:

``
Characterizing the \rR\ relationship with mean and coefficient of variation also helps explain results based on compartmental models [46, 37], because the mean and variance of the generation interval is linked to the mean and variance of latent and infectious periods.
For example, less-variable latent periods result in less-variable generation intervals, whereas less-variable infectious periods result in both less-variable and shorter generation intervals [39].
This explains the apparent anomaly between earlier results: 
when mean generation interval is held constant, less-variable infectious periods only reduces variation in generation intervals, leading to lower \RR;
when mean infectious period is held constant, less-variable infectious periods reduces the overall length of generation intervals, leading to higher \RR\ [46].

There is a simple intuition for this result.
Initial exponential growth of an epidemic is largely driven by shorter infections.
Increasing variation in latent periods results in increase in number of infections with early progression and a faster epidemic -- equivalently, lower \RR\ is required to match the desired value of $r$.
Increasing variation in infectious period results in increase in number of infections with early recovery and a slower epidemic -- equivalently, higher \RR\ is required to match the desired value of $r$
''


\revtext{Page 9 typos. ‘generating a realistic’, ‘data is are limited.’, ‘latent period to a an’}

Fixed

\revtext{Page 10 Suggest “We investigated this our approximation approach using three different examples.”}

\revtext{Figure 3. What is the justification for including $\rho$ > 1?}

While $\rho < 1$ is the relevant range for the Ebola outbreaks, we chose higher $\rho$ to illustrate that the gamma approximation works over a broad range of growth rates in this example.

\revtext{Page 13 “there is small little difference”}

Fixed.

\revtext{Page 14 ‘for rabies’ not ‘for the Rabies’, gamma lower case}

Fixed.

\revtext{Page 15, first paragraph of discussion change to present tense.}

We have followed the reviewer's comment in changing the text.

\revtext{Claim the gamma approximation was introduced in [29] is incorrect as [37] predates it.}

[37] approximates latent and infectious period with gamma distributions; this is different from approximating genertaion-interval distribution with a gamma distribution.

\revtext{The bibliography style should be consistent. A lot of work is needed here}

Done.

\revtext{S1.1 ‘For this model, the generation interval distribution’, ‘the squared coefficient of variation’, ‘relationship under for the SEIR’}

We have followed the reviewer's suggestions here. 

\rev{Reviewer 2}

\revtext{The revised manuscript addresses some of the issues that were raised.}

\revtext{For the original paper, I thought that the main novelty was the
proposal of an estimator of R based on the gamma distribution. This
was incorrect, there were several earlier publications that have
introduced this paper. The authors have now included a literature
review and a useful table (Table 1) to guide the reader through what
is known about the topic.}

\revtext{The authors do not provide any guidance on the generality, other than
stating the approach works well for the examples that are studied.
This is made clear in the revisions.}

We appreciate your comments. We tried to make the contribution of our paper clear in our previous revision.

\rev{Reviewer 3}

\revtext{In this paper authors explore the relationship between R and the
intrinsic growth rate ``r'' under the premises of exponential growth of
epidemics with examples for Ebola, rabies, and measles. While the
theoretical parts of the paper are sound, the practical parts are not
well connected to empirical observations. In particular, the behavior
of an outbreak during the initial phase is an approximation based on a
list of assumptions and conditions.}

\revtext{Notation: Authors denote the basic reproduction number as R while it should be denoted as \Ro.}

Fixed.

\revtext{It should be noted that R0 only applies at time 0 when the epidemic takes off.}

We made this clearer by changing the first sentence of the first paragraph: ``Infectious disease research often focuses on estimating the reproduction number \RR, i.e., the number of new infections caused on average by a single infection, and the related basic reproduction number \Ro, the value of \RR\ for a single primary infection in a fully susceptible population.''

\revtext{Assuming exponential growth is a strong assumption as it
implies that the growth rate r is constant in order to maintain an
unchanged R0 over generations during the initial growth phase (all
individuals carry the same intrinsic growth rate and r is independent
of time). Yet, the ascending growth phase of many outbreaks for
different infectious diseases follow sub-exponential growth rather
than exponential growth as recently noted in several publications
(see e.g.,  Viboud et al. Epidemics 2016; Chowell et al. J Roy Soc
Interface 2016).}

\revtext{After time 0, any systematic changes in the growth rate r will
affect the effective reproduction number R(t), and R0 no longer
applies.  Hence, assuming that epidemics grow with a sustained R0
beyond a couple of generation intervals may be unrealistic. That is,
one has to keep in mind that exponential growth may only be a good
approximation for the very few generations under strong assumption
that R0 is the same for all individuals including the individuals
seeded at generation 0.}

This is a good point. There are strong theoretical reasons to expect early exponential growth, and a long history of studying exponential growth in this context. Nonetheless, recent work has called this approach into question. We now address this point in our Discussion.

\end{document}
